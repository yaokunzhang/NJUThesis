\chapter{引言}
\section{研究动机}

Rust 是一种开源系统编程语言,旨在提供内存安全和线程安全。它由Mozilla Research首次发布于2010年,以解决C和C++语言中常见的内存错误和并发问题。Rust 强调安全性和性能,其内存安全保证主要通过一种称为所有权(ownership)的系统实现,该系统无需垃圾收集器即可管理内存。

作为一门以安全为特点的系统级编程语言,其开发的软件质量备受关注。近年来,随着软件开发的复杂度日益增加,研究者们开始关注软件变更和错误倾向性。例如,在对Java语言的七个不同软件系统进行的一项研究中,通过分析软件的发展快照和构建马尔可夫模型,探索了设计模式和反模式之间的双向变异及其对软件质量的影响 \cite{kermansaravi2021investigating}。该研究发现,设计模式和反模式可以互相转变,某些特定的代码变更类型(如重命名、注释、声明和操作符的更改)主要触发这些模式的变异。更重要的是,这些变异在特定的上下文中更容易导致错误,为软件的设计模式转换和防止引入反模式提供了重要的视角。

另外还有Khomh等人在2012年的一项研究\cite{khomh2012exploratory},该研究深入分析了反模式对面向对象系统中类的变更和错误倾向性的影响。通过在ArgoUML、Eclipse、Mylyn和Rhino等软件系统的54个版本中检测到13种反模式,研究探讨了存在反模式的类与其他类相比,其变更或修复错误的可能性是否更高,这种可能性(如果更高)是否仅仅由于类的大小或反模式的存在,以及哪些类型的变更会影响参与反模式的类。

上述研究的发现对于Rust语言的研究具有启示意义。Rust作为一种注重安全和性能的系统编程语言,其独特的所有权机制和类型系统为软件设计模式的实现和变异提供了新的可能性和挑战。因此,借鉴设计模式与反模式之间变异的研究方法,对Rust语法结构与软件变更及错误倾向性之间的关系进行深入探究,不仅能够帮助开发者更好地理解Rust语言特性,还能够促进Rust软件项目的质量提升,减少错误引入的风险。

通过将设计模式与反模式的研究成果融入到Rust语言的研究框架中,我们可以更全面地理解不同编程语言特性对软件开发过程中设计决策的影响,进而推动软件工程领域的理论和实践发展。

\section{背景知识}
在本节,我们说明与我们研究相关的背景知识。
\subsection{Rust程序设计语言}
Rust程序设计语言是一种现代的系统级编程语言,它具有内存安全性的同时拥有强大的性能,语法简洁明了,同时借鉴了C/C++的部分特性,同时提供了一些新的机制。
Rust保证内存安全的机制是它的所有权规则,所有权具体内容如下:
\begin{itemize}
    \item Rust中的每一个值都有一个被称为其所有者的变量。
    \item 一个值在任一时刻有且只有一个所有者。
    \item 当所有者(变量)离开作用域,这个值将被丢弃。
\end{itemize}

通过所有权系统,Rust 在编译时进行内存管理,避免了常见的内存安全问题,如空指针和数据竞争。所有权系统确保每个值都有唯一的所有者,并在所有者离开作用域时自动释放内存,这种方式避免了显式的内存管理和垃圾回收机制。

除了所有权系统,Rust 还引入了模式匹配、生命周期、trait 等概念。模式匹配是一种强大的模式识别工具,可以用于解构复杂的数据结构和进行流程控制。生命周期则用于指定引用的有效范围,避免悬垂指针的问题。Trait 是 Rust 中的一种抽象机制,类似于其他语言中的接口或抽象类,用于定义共享行为和规范,为Rust面向对象编程提供方便。

\subsection{逻辑回归}
逻辑回归\cite{kleinbaum2002logistic,stoltzfus2011logistic,lavalley2008logistic,sperandei2014understanding},是一种广泛应用于统计和机器学习领域的分类算法,特别适用于二分类问题。它的基本原理是通过数据进行训练,得到一个预测模型,根据给定自变量进行预测事件发生的概率。

它的公式如下:

$π(X1, X2, ..., Xn) = \frac{e^{C0+C1·X1+...+Cn·Xn}} { 1 + e^{C0+C1·X1+...+Cn·Xn}}$

在这里我们说明并解释一下逻辑回归得到的重要参数:
\begin{enumerate}
  \item C值(系数)
      \begin{itemize}
          \item 系数(Coefficient):逻辑回归中的系数(C值)表示当该变量增加一个单位时,对于因变量(例如,事件发生的对数几率)的影响。具体来说,系数的符号(正或负)显示了变量与结果的关系方向,而系数的大小则表明影响的强度。
          \item 正系数:表示随着自变量的增加,事件发生的几率也增加。
          \item 负系数:表示随着自变量的增加,事件发生的几率减少。
      \end{itemize}
  \item p值\cite{andrade2019p}
      \begin{itemize}
          \item  p值:在统计分析中,p值用来衡量观察到的数据结果发生的概率,如果假设检验的零假设为真。在逻辑回归中,p值用来测试每个变量的系数是否显著不同于0。
          \item 显著性阈值:通常,如果p值小于0.05(或其他设定的显著性水平,如0.01),则认为该变量的系数在统计上是显著的,即该变量对模型的影响是重要的,并且不太可能是随机结果。
          \item 高p值:如果p值大于显著性水平(例如大于0.05),则认为该变量的系数在统计上不显著,表明该变量对因变量的影响可能只是偶然的。
      \end{itemize}
\end{enumerate}
  
\subsection{优势比}

优势比(Odds Ratio, OR)\cite{bland2000odds, mchugh2009odds}是一个在统计学中广泛使用的度量,它用来描述在一项研究中某个事件发生的机会与另一个事件发生的机会之间的比例。优势比常用于医学、生物统计、流行病学以及社会科学研究,特别是在病例对照研究(case-control studies)和纵向研究中非常流行。

优势比的计算基于一个四格表(2x2表),其中包含两组数据,每组数据分为两类:事件发生或未发生。例如,在医学研究中,这两组可能是接受特定治疗的患者与未接受治疗的患者,事件则可能是治疗效果的有无。
在这里,我们给出一个示例四格表,如表\ref{tab:odds}:

\begin{table}[h]
	\centering
 	\caption{}
	\begin{tabular}{ccc}
		      & 事件发生 & 事件未发生 \\
		接受治疗  & a    & c     \\
		未接受治疗 & b    & d    
	\end{tabular}

	\label{tab:odds}
\end{table}

对于如上示例,优势比的计算公式是:
\[
\text{OR} = \frac{\frac{a}{c}}{\frac{b}{d}} = \frac{ad}{bc}
\]

\section{相关工作}
变更倾向和错误倾向:
khomh在2012年\cite{khomh2012exploratory}研究了类中反模式与类变更和错误倾向性之间的关系,证实了某些反模式对变更和错误有一定的影响;malhotra在2013年的研究\cite{malhotra2013investigation}中利用贝叶斯网络(BN)模型将面向对象的软件度量与软件错误内容及错误倾向性联系起来,提出了一种全新的方法论,用于分析软件质量。linares在2013年\cite{linares2013api}研究分析了7097款免费Android应用程序所使用的API的错误倾向性和变更倾向性与应用程序成功度(通过用户评分衡量)之间的关系。研究结果揭示了一个关键问题:频繁使用易出错和易变更的API会对应用程序的成功产生负面影响。posnett在2011年\cite{posnett2011empirical}研究代码中容易变更部分的识别方法,通过设计模式和元模式(metapattern)角色的比较,发现类的大小比设计模式或元模式角色更能解释变更倾向。研究结果表明,类的大小是变更倾向的更强决定因素,尽管识别设计模式角色有其他重要用途,但在识别易变类方面,类的大小可能是更好的指标。denaro在2002年\cite{denaro2002deriving}通过使用 logistic 回归,展示了如何建立将软件度量和软件错误倾向联系起来的模型,适用于同质软件产品的类别。它还提出了使用交叉验证来选择有效模型,即使是针对小数据集。这也为我们的研究提供了一定的示范意义。arisholm在2007年\cite{arisholm2007data}试图构建预测模型,以识别 Java 系统中存在高错误概率的部分。singh在2009年\cite{singh2009software}这篇论文中,作者建立了一个支持向量机(SVM)模型,以找到由 Chidamber 和 Kemerer 提供的面向对象度量与错误倾向之间的关系。该研究表明,支持向量机方法也可以用于构建软件质量模型。jaafar在2014年\cite{jaafar2014anti}研究分析了反模式的变异、它们经历的变化以及反模式演化行为与错误倾向之间的关系。
yu在2002年\cite{yu2002predicting}通过经验验证了一组面向对象的度量,以评估其在预测错误倾向方面的有用性。ma在2007年\cite{ma2007statistical}研究了一种使用改进的随机森林算法预测错误模块的方法。kumar在2019年\cite{kumar2019software}研究了通过实施基于遗传算法的机器学习方法预测错误倾向。

Rust语言相关工作:
Rust作为一个近些年刚刚兴起的语言,有关于它的软件质量相关的工作,比较稀少,由于其安全性这一招牌特性,evans在2020年\cite{evans2020rust}在真实世界的Rust库和应用中使用Unsafe Rust进行了大规模的实证研究。研究发现,在Rust库中使用unsafe关键字的比例不到30\%,但超过一半的库因为隐藏在库调用链中的Unsafe Rust,不能完全被Rust编译器静态检查。这表明,尽管使用unsafe关键字的情况有限,但不安全代码的传播对Rust作为一个内存安全语言的声明提出了挑战。此外,为了帮助Rust软件开发者了解何时他们的代码是不安全的,建议对Rust编译器和Rust中心仓库的接口进行一些改变。Rust语言通过其所有权类型系统强制性地管理内存访问和共享,有效预防了常见的编程错误,如内存错误和数据竞争等。然而,这种系统的限制也使得实现某些设计(如需要别名的数据结构)变得困难或不可能。为解决这一限制,Rust允许将代码块声明为unsafe,从而豁免某些类型系统的限制。astrauskas在2020年\cite{astrauskas2020programmers}通过分析大量Rust项目,实证研究了unsafe代码的实际使用情况,以评估Rust假设的有效性和分类unsafe代码的用途。研究发现,尽管大多数unsafe代码都简单且封装良好,但在需要与其他语言互操作时,广泛使用了unsafe功能。这项研究部分支持Rust假设,即程序员应谨慎使用unsafe代码,确保其易于审查,并通过安全抽象进行封装,使得客户端代码能够安全编写。pinho在2019年\cite{pinho2019towards}分析了在航空航天和其他安全关键领域采用Rust编程语言的潜在优势。通过与C语言的安全编程比较,研究表明Rust可以缓解C语言安全编程中的一些问题。这项研究旨在展示Rust编程语言强大的安全支持,以便更清晰地评估Rust在安全关键系统中的应用价值。bae在2021年\cite{bae2021rudra}报告了RUDRA,一个用于分析和报告不安全Rust代码中潜在内存安全漏洞的程序。zhu在2022年\cite{zhu2022learning}研究了如下问题:(1) 哪些安全规则在学习和编程时具有挑战性;(2) 在哪些情况下,应用某条安全规则会变得更困难;(3) Rust编译器在调试安全规则违规时是否足够有帮助。zhang在2022年\cite{zhang2022towards}通过与C语言的对比来理解Rust的性能。这项研究表明,尽管Rust在某些情况下表现出性能开销,但其设计带来的内存安全性和其他优势使其在系统编程中具有很高的应用价值。ardito在2020年\cite{ardito2020rust}研究设计了一个库,能够提取包括Rust在内的十种不同编程语言的十一项可维护性度量。


\section{本文研究结构}
本文一共有六章,各章节内容和结构如下:

第一章:引言。介绍了我们的研究背景,从国内外对于软件质量的研究,以及当下Rust语言的现状,引出我们的实证研究。最后介绍本文的研究结构。

第二章:研究意义。在这一章,我们详细分析了我们的研究可能具有的意义和价值。

第三章:研究背景和国内外研究现状。介绍了研究相关的背景知识,给读者进行简单的介绍,另外介绍了当下的研究现状,让读者对当前的研究领域有一个基本的了解。

第四章:研究问题和方法设计。介绍我们的研究问题,我们给出了四个贴合我们实证研究主题的研究问题,这几个问题体现了我们研究的主要内容。介绍研究问题的自变量部分和因变量部分,它们是我们从github仓库中获取到的一些变量信息,我们研究二者之间的关系。介绍了数据收集,给出了实验中各种必需的实验数据的具体收集方式。介绍了研究方法,给出了对获取到的数据处理的方法,从这一步骤我们获得了结果数据。

第五章:研究结果。针对每个研究问题,我们给出了对于相应结果数据的初步结论。

第六章:讨论。针对每个研究问题,我们基于上一章得到的初步结论进行评估,并分析现研究的不足,最后,展望未来的工作。

第七章:研究总结。该部分总结了本实证研究,给出了研究结论,并分析了目前的问题,对未来的研究提出了意见。

