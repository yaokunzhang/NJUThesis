\chapter{评估与不足}
在本研究中,我们通过对四个问题的讨论,得出了一定的结论,下面我们对四个问题的结论进行评估:
\begin{itemize}
    \item 研究问题一:我们通过实验分析出部分语法项具有变更倾向,部分不具有变更倾向,还有一小部分因输入数据少,导致无法得出结论。
    \item 研究问题二:我们通过实验分析出少部分语法项的某些成分对语法项的变更倾向具有促进作用或抑制作用,较多的语法项无法通过逻辑回归得到良好的拟合,因此无法得出结论。
    \item 研究问题三:我们通过实验分析出部分语法项具有引入错误倾向,部分不具有引入错误倾向,还有一小部分因输入数据少,导致无法得出结论。
    \item 研究问题四:我们通过实验分析出少部分语法项的某些成分对语法项的引入错误倾向具有促进作用或抑制作用,较多的语法项无法通过逻辑回归得到良好的拟合,因此无法得出结论。
\end{itemize}

总体来看,研究问题一三实验结果比较明显,得出了一些符合研究主题的结论。而研究问题二四受数据量少的限制,研究方案的不足影响,无法得出良好的具有普遍性的结论。

对于实验的不足,我们有如下的分析:
\begin{enumerate}
    \item 研究尺度不够精细:以问题三为例,我们统计错误引入的单位是一个commit,即一个commit中只要引入了造成错误的代码,我们就认为此commit引入了错误,但是事实上,在版本控制系统中,一个commit的改变很可能是很大篇幅的,引入错误可能只占改变内容的一部分,由于我们尺度的不够精细,我们会将实际未引入错误的部分也视作引入了错误。
    \item 获取到的bug-fix commit不够准确:我们获取的方法是从Github Issue中Issue的标签中寻找匹配目标标签的commit,实际上可能疏忽了一些修复bug的标签,进而导致某些bug-fix commit未被统计到。
    \item 树编辑距离算法的局限性:树编辑距离算法虽然相较于传统的求解文本比对算法在准确度方面有很大程度的提升,但是在不同的权重机制下,算法的求解结果仍然可能不唯一,从而产生不同的解,导致我们的实验获取代码diff时并不一定获得最精确的结果。
    \item 选取的实验对象(github仓库)较少:目前我们实验的对象只有12个仓库,针对每一特定领域更显不足,在特定领域方面得出的结论难以具有说服力。
    \item 分析方法的不足:例如我们选取的逻辑回归算法在本实验中在很多语法项上并没有获得很好的拟合效果。我们没有考虑好研究问题与研究方法二者的相互需要,导致没有找到更好的方案。
\end{enumerate}

我们认为,我们的工作对于后续的工作具有很大的指导意义,具体来说,我们未来的研究工作可以包括以下一些方向:

\begin{itemize}
    \item 扩展研究样本:我们目前的研究可能主要集中在特定的项目或者数据集上,未来可以将研究扩展到更多的Rust项目,包括开源和商业项目,以增强研究结果的普适性和可靠性。
    \item 采用更精细的研究方法:设计更好的算法,识别bug-fix commit和错误引入commit,针对具体语法成分,设计更合理有效方案识别
,采用更好的统计分析办法,分析各种语法成分对于变更或是错误倾向的影响力。
    \item 深入分析语法特性:进一步研究不同的Rust语法特性如何具体影响软件的变更倾向和错误倾向,例如探索模式匹配、所有权和借用机制等Rust特有语法的影响,这些在我们现有研究中还未体现。
    \item 比较研究:将Rust与其他语言(如C++、Go等)进行比较,探索不同语言特性对软件变更和错误倾向的影响,在语言层面上横向分析Rust在系统安全性和可维护性方面的优势和不足。
    \item 实时监控和预测模型:开发基于实时数据的监控系统,预测Rust程序的潜在错误和变更需求,帮助开发者及时优化代码,减少生产环境中的风险。
    \item 工具和插件开发:基于研究成果,开发集成到Rust开发环境中的工具和插件,如代码质量检测工具、风险提示插件等,提升Rust程序的开发效率和安全性。
\end{itemize}